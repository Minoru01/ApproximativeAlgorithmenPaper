Bei der Beobachtung von Streams können unter Umständen sehr große Datenmengen auftreten. Das Speichern aller dieser Daten stellt sich oft nicht als tragbar heraus. Trotzdem wäre dies nötig, um gewisse Kennzahlen wie unterschiedliche Elemente oder die am häufigst auftretenden Elemente zu identifizieren. Da jedoch durch die Datenmenge solche Probleme exakt nur selten effizient gelöst werden können, wird eine Alternative hierfür benötigt. Es stellt sich heraus, dass Näherungslösungen meist für ausreichend befunden werden, insbesondere, weil nur leichte Einbußen der Genauigkeit zu einer stark gesteigerten Performanz führen können. Für die unterschiedlichsten Kennzahlen, die aus einem solchen Stream entnommen werden können, wurden diverse approximative Algorithmen entwickelt, welche diese Probleme annähernd korrekt lösen sollen.