\section{Proof of Concept}



\begin{equation}
	\frac{n}{\frac{1}{x_1}+\frac{1}{x_2}+\ldots+\frac{1}{x_3}}=\frac{n}{\displaystyle\sum_{i=1}^{n}\frac{1}{x_i}}=\left(\frac{\displaystyle\sum_{i=1}^{n}x_i^{-1}}{n}\right)^{-1}
	\label{eq:harmonic-mean}
\end{equation}

\autoref{eq:harmonic-mean}



\begin{equation}
	\begin{aligned}
		e& = \frac{1.04}{\sqrt{m}} \\
		m& = \left(\frac{1.04}{e}\right)^2 = 2^b \\
		b& = \log_2 m = \log_2 \left(\frac{1.04}{e}\right)^2
	\end{aligned}
	\label{eq:target-error}
\end{equation}

Der benötigte Speicherplatz -- also die Anzahl an benötigten Registern und führender Bits -- lässt sich wie in \autoref{eq:target-error} dargestellt berechnen. So benötigt die in unserem Beispielprogramm gewählte Genauigkeit von einem Prozent $b = log_2 \left(\frac{1.04}{0.01}\right)^2 = 13.40$, beziehungsweise aufgerundet die 14 führenden Bits des Hashes. Dies führt zu $m = 2^{14} = 16384$ benötigten Registern und somit einem maximalen Speicherverbrauch von 16\,384 Byte.