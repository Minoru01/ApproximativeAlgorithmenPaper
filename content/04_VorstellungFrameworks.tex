\section{Vorstellung: Framworks}
Im Folgenden werden Frameworks vorgestellt die approximative Algorithmen für die Streamverarbeitung anbieten.

Das Apache Spark Framework bietet eine Bibliotheke für die Verarbeitung von Streams in Java, Scala und Python. 
In diesem Framework stehen seit der Version 2.0 unter anderem auch die Funktionen \texttt{approxCountDistinct} und \texttt{approxQuantile} zur Verfügung. 
\texttt{approxCountDistinct} nutzt den HyperLogLog++ Algorithmus um eine ungefähre Lösung für das ihm gegebene Problem zu liefern \cite{hunter2016}.
Hierbei kann mit \texttt{approxCountDistinct} für die Daten eines Streams die ungefähre Anzahl unterschiedlicher Worte 
und mit \texttt{approxQuantile} ein ungefährer Wert zu einer bestimmten Quantile ausgegeben werden \cite{hunter2016}. 

\todo[inline]{quelle von apache spark approxQuantile überprüfen %https://spark.apache.org/docs/2.1.0/api/R/approxQuantile.html
%http://dx.doi.org/10.1145/375663.375670
}

Implementierungen verbreiteter approximativer Algorithmen findet man in einigen weiteren Bibliotheken. 
Ein Beispiel für eine solche Bilbiothek ist stream-lib.
stream-lib bietet unter anderem die Algorithmen LogLog, 
HyperLogLog und HyperLogLog++ für die Berechnung von Kardinalitäten in Streams, 
sowie die Algorithmen T-Digest und Q-Digest für die Berechnung von Quantilen in Streams. 

%https://github.com/addthis/stream-lib

\todo[inline]{Apache Spark Dokumentation zur Hilfe nehmen und nachschauen welche Algorithmen alles angeboten werden und ggf mit Algorithmen anderer Anbieter vergleichen}