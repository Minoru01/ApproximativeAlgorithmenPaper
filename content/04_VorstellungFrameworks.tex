\section{Vorstellung: Framworks und Bibliotheken}
Im Folgenden werden Frameworks und Bibliotheken vorgestellt die approximative Algorithmen für die Streamverarbeitung anbieten. 

\subsection{Apache Spark}
Das Apache Spark Framework bietet eine Bibliotheke für die Verarbeitung von Streams in Java, Scala und Python. 
In diesem Framework stehen seit der Version 2.0 unter anderem auch die Methoden \texttt{approxCountDistinct} und \texttt{approxQuantile} zur Verfügung. 
\texttt{approxCountDistinct} nutzt den HyperLogLog++ Algorithmus um eine ungefähre Lösung für das ihm gegebene Problem zu liefern \cite{hunter2016}.
Die Methode kann die ungefähre Anzahl unterschiedlicher Worte für die Daten eines Streams ermitteln.
\texttt{approxQuantile} nutzt den Greenwald-Khanna Algorithmus \cite{greenwald2001} und kann ein ungefährer Wert zu einer bestimmten Quantile ausgeben.

%https://spark.apache.org/docs/2.1.0/api/R/approxQuantile.html
%http://dx.doi.org/10.1145/375663.375670



\subsection{Algebird}


\subsection{Apache DataSketches}


\subsection{stream-lib}
Die öffentliche stream-lib Bibliotheke bietet für Java unter anderem die approximativen Algorithmen LogLog, 
HyperLogLog und HyperLogLog++ für die Berechnung von Kardinalitäten in Streams, 
sowie die Algorithmen T-Digest und Q-Digest für die Berechnung von Quantilen in Streams. 
Das Github-Projekt auf dem die Bibliotheke basiert wurde archiviert, es ist also warscheinlich, dass das Projekt von Seiten des Entwicklerteams nicht weiterentwickelt wird. 
%https://github.com/addthis/stream-lib
