\section{Vorstellung: Algorithmen}
Im Folgenden werden approximative Algorithmen vorgestellt die in der Streamverarbeitung nutzbringend eingesetzt werden können.

\subsection{StreamApprox}
StreamApprox ist ein Algorithmus der dafür entwickelt wurde, 
die Datenmenge eines Streams auf eine bestimmte Größe zu limitieren, 
indem er nur einige der Daten des Streams zur weiterreicht. 
Dabei erfolgt die Selektierung welche der Daten weiter gegeben werden durch einen Zufallsalgorithmus. 
Auf diese Weise wird das Ergebnis der Auswertung des Streams weniger stark beeinträchtigt, 
als wenn die Daten, die die Verarbeitungskapazitäten überschreiten, 
überhaupt nicht betrachtet werden. 
Der Algorithmus ist dabei sowohl auf Batch-, als auch auf Pipeline-basierte Streams anwendbar.
Das besondere an StreamApprox ist die Fähigkeit, 
die Größe des Streams zu limitieren und dabei so wenig Daten wie möglich aus dem Stream zu entfernen. 
Der Algorithmus reduziert also nur dann den Datenstrom, wenn dieser ein eingestelltes Limit überschreitet. 
Diese Fähigkeit übersteigt die von random sampling Algorithmen, 
die für eine ähnliche Ausgabe die absolute Anzahl an Elementen des Streams kennen müssten. 
Ermöglicht wird das durch die Verwendung von reservoir Sampling, 
bei dem die Elemente des Streams in einen Pufferspeicher geschrieben werden, 
wobei neue Elemente an einer zufälligen Stelle des Puffers den vorherigen Wert überschreiben. \cite{quoc2017} 

Dieses Vorgehen kann hilfreich sein, 
wenn die Datenmenge für die gewünschte Analyse stark varriert oder unnötig groß ist und reduziert werden soll, 
um eine bestimmte Verarbeitungsgeschwindigkeit beizubehalten, 
ohne das Ergebnis der Analyse zu stark zu beeinflussen.
Es kann ebenfalls hilfreich sein, 
wenn aufgrund von fehlender Ressourcen zur Verarbeitung die Menge an gelieferten Daten reduziert werden soll.