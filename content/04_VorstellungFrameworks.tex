\section{Vorstellung: Frameworks und Bibliotheken}
Die vorgestellten und weitere approximative Algorithmen sind bereits in einigen Frameworks und Bibliotheken eingebunden. Nachfolgend werden drei solche vorgestellt, um einen kurzen Überblick über die eingebundenen Algorithmen sowie die Verwendung der Frameworks und Bibliotheken zu bieten.

\subsection{Apache Spark}
Das Apache Spark Framework bietet Bibliotheken für Streaming, SQL, Machine Learning und Graphverarbeitung in Java, Scala und Python an. 
Für SQL-Anfragen stehen dabei seit der Version 2.0 unter anderem auch die Methoden \texttt{approxCountDistinct} 
und \texttt{approxQuantile} zur Verfügung. 
\texttt{approxCountDistinct} nutzt den Hyper\-LogLog++-Algorithmus, 
um eine ungefähre Lösung für das ihm gegebene Problem zu liefern \cite{hunter2016}.
Die Methode kann die ungefähre Anzahl unterschiedlicher Worte für die Daten einer Datenbank ermitteln.
\texttt{approxQuantile} nutzt den Greenwald-Khanna-Algorithmus \cite{greenwald2001} 
und kann einen ungefähren Wert zu einem bestimmten Quantil ausgeben.
In der Streaming Bibliothek werden Implementierungen des Bloom Filters für contains-Abfragen und Count-Min-Sketch für Abfragen der Häufigkeiten von Elementen angeboten.

\subsection{Apache DataSketches}
Apache DataSketches ist eine Bibliothek für die Sprachen Java, C++ und Python, die stochastische Streamingalgorithmen, 
auch Sketches genannt, anbietet. 
Diese Sketches können auch sehr große Mengen an Daten aus Streams verarbeiten 
und liefern dabei approximative Lösungen. 
In welcher Größenordnung sich die Abweichung der Lösung befinden wird kann dabei mathematisch bestimmt 
und vor dem Einsatz in einem Projekt einkalkuliert werden.

Neben dem bereits bekannten HyperLogLog-Algorithmus bietet DataSketch auch den Compressed Probabilistic Counting (CPC) Algorithmus, 
für eine noch höhere Präzision bei gleichem Speicherbedarf an.
Ebenfalls implementiert ist das Theta Sketch Framework, 
mit dem zum Beispiel der Wert des k-ten Minimums ermittelt werden kann.
Die Bibliothek umfasst auch Algorithmen, mit denen Quantile und die häufigsten Elemente eines Streams ermittelt werden können, 
sowie die Algorithmen Reservoir- und Weighted-Sampling, mit denen die Datenmenge von Streams reguliert werden können.
Die Apache DataSketches Bibliothek ist für die Sprachen Java, C++ und Python verfügbar.

\subsection{stream-lib}
Die öffentliche Java-Bibliothek stream-lib bietet Implementierungen für die approximativen Algorithmen LogLog, 
HyperLogLog und HyperLogLog++ für die Bestimmung der Kardinalität von Elementen in Streams, 
sowie die Algorithmen T-Digest und Q-Digest für die Bestimmung von Quantilen in Streams \cite{streamlib2019}. 
Das Projekt befasst sich, anders als Apache Spark, speziell mit der Implementierung approximativer Algorithmen und ist minimalistischer umgesetzt als die Implementierungen von Apache DataSketches. Deshalb bietet sich das Projekt besonders als Unterstützung beim Einlernen in das Thema approximativer Algorithmen oder der Einbindung der angebotenen Algorithmen in eigene Projekte an.
Das GitHub-Projekt, auf dem die Bibliothek basiert, wurde archiviert, 
es ist also warscheinlich, dass das Projekt von Seiten des Entwicklerteams nicht weiterentwickelt wird.