\section{Vorstellung: Frameworks und Bibliotheken}
Die vorgestellten Algorithmen und einigen weitere approximative Algorithmen wurden bereits in einige Frameworks und Bibliotheken eingebunden. Nachfolgend werden deshalb einige dieser vorgestellt, um einen kurzen Überblick über deren Einbindung, sowie die verwendung der Frameworks und Bibliotheken zu bieten.

\subsection{Apache Spark}
Das Apache Spark Framework bietet Bibliotheken für Streaming, SQL, Machine Learning und Graphverarbeitung in Java, Scala und Python an. 
Für SQL-Anfragen stehen dabei seit der Version 2.0 unter anderem auch die Methoden  \texttt{approxCountDistinct} 
und \texttt{approxQuantile} zur Verfügung. 
\texttt{approxCountDistinct} nutzt den HyperLogLog++ Algorithmus, 
um eine ungefähre Lösung für das ihm gegebene Problem zu liefern \cite{hunter2016}.
Die Methode kann die ungefähre Anzahl unterschiedlicher Worte für die Daten eines Streams ermitteln.
\texttt{approxQuantile} nutzt den Greenwald-Khanna Algorithmus \cite{greenwald2001} 
und kann ein ungefährer Wert zu einer bestimmten Quantile ausgeben.
In der Streaming Bibliothek werden Implementierungen des Bloom Filters für contains-Abfragen und Count-Min Sketch für Abfragen der Häufigkeiten von Elementen angeboten.

%https://spark.apache.org/docs/2.1.0/api/R/approxQuantile.html
%http://dx.doi.org/10.1145/375663.375670

\subsection{Apache DataSketches}
Apache DataSketches ist eine Bibliothek für die Sprachen Java, C++ und Python die stochasitische Streamingalgorithmen, 
auch Sketches genannt anbietet. 
Diese Sketches können auch sehr große Mengen an Daten aus Streams verarbeiten 
und liefern dabei approximative Antworten. 
In welcher Größenordnung sich die Abweichung befinden wird kann dabei mathematisch bestimmt 
und vor dem Einsatz in einem Projekt einkalkuliert werden.

Neben dem bereits bekannten HyperLogLog Algorithmus bietet DataSketch auch den Compressed Probabilistic Counting (CPC) Algorithmus, 
für eine noch höhere Präzision bei gleichem Speicherbedarf an.
Ebenfalls implementiert ist das Theta Sketch Framework, 
mit dem zum Beispiel der Wert des k-ten Minimums ermittelt werden kann.
Die Bibliotheke umfasst auch Algorithmen mit denen Quantilen und die häufigsten Elemente eines Streams ermittelt werden können, 
sowie die Algorithmen reservoir- und weighted-Sampling, mit denen die Datenmenge von Streams reguliert werden können.
Die Apache DataSketches Bibliothek ist für die Sprachen Java C++ und Python verfügbar.

\subsection{stream-lib}
Die öffentliche stream-lib Bibliotheke bietet für Java unter anderem die approximativen Algorithmen LogLog, 
HyperLogLog und HyperLogLog++ für die Bestimmung von Kardinalitäten in Streams, 
sowie die Algorithmen T-Digest und Q-Digest für die Bestimmung von Quantilen in Streams. 
Das Github-Projekt auf dem die Bibliotheke basiert wurde archiviert, 
es ist also warscheinlich, dass das Projekt von Seiten des Entwicklerteams nicht weiterentwickelt wird. 
%https://github.com/addthis/stream-lib
