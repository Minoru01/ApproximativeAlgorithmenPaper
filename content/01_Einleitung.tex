\section{Einleitung}
Was sind approximative Algorithmen?\\




Das Apache Flink \cite{quoc2019}

Das Apache Spark Framework bietet eine Bibliotheke für die Verarbeitung von Datenströmen in Java, Scala und Python. 
Dafür stehen seit der Version 2.0 unter anderem auch die Funktionen \texttt{approxCountDistinct} und \texttt{approxQuantile} zur Verfügung. 
Beide Funktionen nutzen den Hyperloglog++ Algorithmus um eine ungefähre Lösung für das ihnen gegebene Problem zu liefern \cite{hunter2016}.
Hierbei kann mit \texttt{approxCountDistinct} für die Daten eines Streams die ungefähre Anzahl unterschiedlicher Worte 
und mit \texttt{approxQuantile} ein ungefährer Wert zu einer bestimmten Quantile ausgegeben werden \cite{hunter2016}. 

Implementierungen verbreiteter approximativer Algorithmen findet man in einigen weiteren Bibliotheken. 
Ein Beispiel für eine solche Bilbiothek ist stream-lib.
stream-lib bietet unter anderem die Algorithmen Loglog, 
Hyperloglog und Hyperloglog++ für die Berechnung von Kardinalitäten in Streams, 
sowie die Algorithmen T-Digest und Q-Digest für die Berechnung von Quantilen in Streams. 

%https://github.com/addthis/stream-lib
import für Loglog hyperloglog und hyperloglog++ unter Java:
\texttt{com.clearspring.analytics.stream.cardinality.Class LogLog}
\texttt{com.clearspring.analytics.stream.cardinality.HyperLogLog}
\texttt{hyperloglog++ com.clearspring.analytics.stream.cardinality.HyperLogLogPlus}


\todo[inline]{Apache Spark dokumentation zur hilfe nehmen und nachschauen welche algorithmen alles angeboten werden und ggf mit algorithmen anderer anbieter vergleichen}

\todo[inline]{über vorläufige Datenexploration (preliminary data exploration) informieren}