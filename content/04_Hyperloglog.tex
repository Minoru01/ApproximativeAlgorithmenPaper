\section{Test: Hyperloglog}
In diesem Kapitel wird der Hyperloglog Algorithmus getestet indem er von Wikipedia zur verfügung gestellte Datenstreams verarbeitet. Das Ergebnis dieser verarbeitung wird Evaluiert. 

Hyperloglog ist eine Verbesserung des Loglog-Algorithmus von Durand und Flajolet \cite{durand2003}. Der Hyperloglog-Algorithmus wurde von Flajolet et al. \cite{flajolet2007} entwickelt, um die ungefähre Anzahl verschiedener Elemente in einer sehr großen Datenmenge zu ermitteln. 

In practice, the HYPERLOGLOG program is quite efficient: like the standard LOGLOG of [10] or
MINCOUNT of [16], its running time is dominated by the computation of the hash function, so that it is
only three to four times slower than a plain scan of the data (e.g., by the Unix command “wc -l”, which
merely counts end-of-lines \cite{flajolet2007}

As a final summary, the algorithm proves to be easy to code and efficient, being even nearly optimal
under certain criteria. On “real-life” data, it appears to be in excellent agreement with the theoretical analysis,
a fact recently verified by extensive tests (see Figure 5 for a sample) conducted by Pranav Kashyap,
whose contribution is here gratefully acknowledged.
The program can be applied to very diverse collections
of data (only a “good” hash function is needed), and, once duly equipped with corrections, it can
smoothly cope with a wide range of cardinalities–from very small to very large. In addition, it parallelizes
or distributes4 optimally and can be adapted to the “sliding window” usage \cite{flajolet2007}

hyperloglog ist eine verbesserung von loglog \cite{flajolet2007}

mögliche quelle: \\
Ursprung \cite{flajolet2007},
Hyperloglog++ verbesserung durch google \cite{heule2013},
außerdem siehe auskommentiert hier:
%https://databricks.com/de/blog/2016/05/19/approximate-algorithms-in-apache-spark-hyperloglog-and-quantiles.html