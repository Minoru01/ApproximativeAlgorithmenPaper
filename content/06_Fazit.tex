\section{Fazit}

Die in diesem Paper untersuchten Algorithmen zur Approximation schwer exakt zu lösender Probleme haben sich als sehr genau herausgestellt.
Es wurde jeweils eine sehr starke Verbesserung der Performanz mit nur geringen Einbußen der Genauigkeit des Ergebnisses festgestellt.

Jedoch decken alle Algorithmen nur einen für eine einzelne Aufgabe spezifischen Themenbereich ab.
Somit ist es notwendig, sollte man mehrere unterschiedliche Kennzahlen untersuchen wollen, mehrere dieser Algorithmen gleichzeitig einzubinden, was gewissen Verbesserungen der Geschwindigkeit oder des Speicherbedarfs negativ entgegenwirkt.

Der HyperLogLog-Algorithmus hat sich im Rahmen des Proof of Concept, insbesondere bei großen Datenmengen, als sehr genau herausgestellt, was ihn zu einer sehr attraktiven Lösung für das Count-Distinct-Problem macht.
Dies zeigt sich auch an der Entwicklung des Algorithmus, da dieser seit vielen Jahren stetig weiterentwickelt und verbessert wird.

Dieser beim HyperLogLog festgestellte Trend wurde in mehreren Bereichen festgestellt.
Viele der Algorithmen werden stetig weiterentwickelt, statt von neuen abgelöst zu werden.
Dies lässt darauf schließen, dass viele aktuelle Algorithmen eine beinahe mathematisch optimale Lösung darstellen, die gegebenen Probleme mit möglichst hoher Genauigkeit zu lösen, ohne dabei zu viel Performanz einzubüßen.

Im Bereich des Streamings bieten sich Bibliotheken wie Apache DataSketches sehr stark an, da für viele der untersuchbaren Kennzahlen hier bereits approximative Algorithmen implementiert sind.
So wird einem Nutzer ein einfacher Zugriff auf diese Näherungswerte in den untersuchten Streams gewährt.