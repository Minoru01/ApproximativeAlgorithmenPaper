\section{Beispiel}

\todo[inline]{k-Means-Algorithmus möglicherweise als weiterer algorithmus}
\todo[inline]{q-digest-Algorithmus möglicherweise als weiterer algorithmus}
\todo[inline]{Streaming Systeme wie Flink , Spark und Apex untersuchen ob diese approximative algorithemn anbieten.}
\todo[inline]{überprüfen ob kombination von approxi und inkrementeller verarbeitung etwas für das paper ist ''IncApprox''(vielleicht im ausblick)(mögliches paper hier im todo als kommentar)
%https://link.springer.com/referenceworkentry/10.1007%2F978-3-319-63962-8_151-1
}
\todo[inline]{begriff Sketch, one-touch processing und approximate Query Processing recherchieren}
\todo[inline]{count Unique: neben Hyperloglog gibt es noch noch Theta Sketch und Compressed Probabilistic Counting (CPC)}
\todo[inline]{sketch für most frequent: items heavy hitters}
\todo[inline]{count min sketch als algorithmus einbauen!}


%\todo[inline]{Todo}

\cite{korte2018}
