\section{Einsatzgebiet Streamverarbeitung}
Bei der Streamverarbeitung treten einige Anforderungen auf, die es für den Einsatz von Technologien zu berücksichtigen gilt. 
So treffen die Daten in einem Stream beispielsweise kontinuierlich ein und sollten abgearbeitet werden bevor die nächsten Daten eintreffen. 
Außerdem sollen die aus dem Stream gewonnenen Informationen möglichst in Echtzeit abrufbar bleiben.
Auch die Datenmenge in einem Stream kann stark variieren, wodurch ein hohes Maß an Skalierbarkeit gefordert werden kann.

Unter diesen Anforderungen bieten sich approximative Algorithmen an, denn sie bieten eine schnellere Ausführungszeit und eine bessere Skalierbarkeit als exakte Algorithmen. 
Für diesen Vorteil büßen sie jedoch ihre absolute Genauigkeit ein. 
In der Praxis überwiegt der Gewinn von Geschwindigkeit dem Verlust an Genauigkeit jedoch häufig \cite{Maas2019}. 





Die Datenverarbeitung in einem Stream unterliegt drei Eigenschaften, die sich gegenseitig beeinflussen. \cite{Maas2019}

\begin{itemize}
\item
Der Genauigkeit der erzeugten Ergebnisse,
\item
der Ausführung in Echtzeit (Performanz)
\item
und der zu verarbeitende Datenmenge
\end{itemize}